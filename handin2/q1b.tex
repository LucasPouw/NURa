\subsection{Q1b}

\subsubsection{Regular method}

We use rejection sampling with a horizontal line at the maximum of $p(x)$ (calculation in next subsubsection).
The histogram is normalized by dividing by the bin width and by the total number of samples in the histogram.

\begin{figure}[H]
    \centering
    \includegraphics[width=\textwidth]{plots/my_solution_1b.png}
    \caption{The radial distribution of galaxies. The analytical distribution (red line) matches the histogram
    of $10^4$ samples well. There are very few samples below $x \approx 10^{-2}$, because the probability of
    drawing a sample here with $10^4$ total samples is too low.
    }
\end{figure}


\subsubsection{A more efficient method}
Regular rejection sampling has a high chance of rejecting samples, because the rejection region between the
horizontal line and $p(x)$ is big. It would be better to do inverse-transform sampling of $p(x)$, but the
CDF does not have a closed-from solution (so that would require `scipy.special` stuff).

Therefore, we will find a proposal function that matches $p(x)$ better than the horizontal line. We need
a function $f(x) > p(x) \, \forall x$ of which we have an analytical CDF that 
can be inverted. We can then do inverse transform sampling on $f(x)$ followed by rejection sampling of $p(x)$.

Looking at the inequality
\begin{equation}
    f(x) \geq 4\pi A b^{3-a} x^{a-1} \exp\left\{-\left(\frac{x}{b}\right)^{c} \right\}
\end{equation}
we first analyze the behavior of $p(x)$ at high $x$ and low $x$. For $x$ approaching zero:
\begin{equation}
    \lim_{x \downarrow 0} \left[ x^{a-1} \exp\left\{-\left(\frac{x}{b}\right)^{c} \right\} \right] = x^{a-1} \, ,
\end{equation}
which means that the power law dominates and we get
\begin{equation}
    p(x) \leq 4\pi A b^{3-a} x^{a-1} \, .
\end{equation}
A first guess for our function is $f(x) = x^{a-1}$, but the rejection sampling will become very inefficient for large $x$, because for $x$ approaching infinity:
\begin{equation}
    \lim_{x \to \infty} \left[ x^{a-1} \exp\left\{-\left(\frac{x}{b}\right)^{c} \right\} \right] = 0 \, ,
\end{equation}
which means the exponential decay dominates $p(x)$. Therefore, we also need $f(x)$ to decay to 0 for large $x$. We can divide by a function of the form $g(x) = 1+\beta x^\gamma$, where $\gamma > a-1$. Using this prescription, $g(x) \ll 1$ for small $x$, such that $f(x)$ approaches $x^{a-1}$ again. At the same time, $g(x) \gg x^{a-1}$ for large $x$, causing a decay to 0. This decay will not be faster than the exponential decay of $p(x)$, so the whole function will still be larger than $p(x)$. In conclusion, our functional form becomes
\begin{equation}
    f(x) = 4\pi A b^{3-a} \frac{x^{a-1}}{1+\beta x^\gamma} \, .
\end{equation}

We now determine $\beta$ and $\gamma$. A natural choice for $\gamma$ comes up when evaluating the integral of $f(x)$:
\begin{equation}
    I(x) = 4\pi A b^{3-a} \int_{0}^{x} \frac{x^{a-1}}{1+\beta x^\gamma} \dd x \,
\end{equation}
which has an analytical solution for $\gamma = a$, because substituting $u = 1 + \beta x^a$, $\dd u = \beta a x^{a-1}$ gives 
\begin{align}
    I(x) 
    &= \frac{4\pi A b^{3-a}}{\beta a} \int_{1}^{1+\beta x^a} \frac{1}{u} \dd u \\
    &= \frac{4\pi A b^{3-a}}{\beta a} \ln \left( 1 + \beta x^a \right) \, .
\end{align}
Now that we're here, we can already calculate the inverse that we want to sample:
\begin{equation}
    I^{-1}(X) = \left[ \frac{1}{\beta} \left( \exp\left\{ \frac{\beta a}{4\pi A b^{3-a}} X \right\} - 1 \right) \right]^{1/a} \, ,
\end{equation}
where $X \sim U(0,1)$.

Almost there. We can choose $\beta$ such that the maxima of $p(x)$ and $f(x)$ coincide, such that $f(x)$ is as close as possible to $p(x)$. We find the location of the maxima by setting the derivative to zero:
\begin{align}
    f'(x) = \frac{x^{-2 + a} (-1 + a - \beta x^a)}{(1 + \beta x^a)^2} &= 0 \\
    x^{-2 + a} (-1 + a - \beta x^a) &= 0 \\
    -1 + a - \beta x^a &= 0 \\
    x = \left( \frac{a-1}{\beta} \right)^{1/a} \, ,
\end{align}
which gives a maximum of $f(x)$ of 
\begin{equation}
    f_{\mathrm{max}} = \frac{4\pi A b^{3-a}}{a} \left( \frac{a-1}{\beta} \right)^{1-1/a} \, .
\end{equation}

Now for $p(x)$:
\begin{align}
    p'(x) = \exp\left\{ -\left(\frac{x}{b}\right)^c\right\} x^{-2 + a} \left(-1 + a - c \left(\frac{x}{b}\right)^c \right) &= 0 \\
    -1 + a - c \left(\frac{x}{b}\right)^c &= 0 \\
    x = b\left(\frac{a - 1}{c}\right)^{1/c} \, ,
\end{align}
giving a maximum of
\begin{equation}
    p_{\mathrm{max}} = 4\pi A b^{2} \left(\frac{a - 1}{c}\right)^{(a-1)/c} \exp\left\{\frac{1-a}{c} \right\} \,
\end{equation}
and then for $a=2.4$, $b=0.25$ and $c=1.6$ we get
\begin{align}
    f_{\mathrm{max}} &= p_{\mathrm{max}} \\
    \frac{b^{1-a}}{a} \left( \frac{a-1}{\beta} \right)^{1-1/a} &= \left(\frac{a - 1}{c}\right)^{(a-1)/c} \exp\left\{\frac{1-a}{c} \right\} \\
    \beta &= (a-1)\left[ \frac{a}{b^{1-a}} \left(\frac{a - 1}{c}\right)^{(a-1)/c} \exp\left\{\frac{1-a}{c} \right\} \right]^{a/(1-a)} \\
    \beta &\approx 47.6114 \, .
\end{align}

\begin{figure}[H]
    \centering
    \includegraphics[width=\textwidth]{plots/q1b_reject_and_inverse_transform.png}
    \caption{Radial galaxy distribution using our improved method. Our proposal distribution (red line) matches our target distribution
    (blue line) more closely than a simple horizontal line. Samples from the proposal distribution are drawn using inverse transform 
    sampling, requiring analytical expressions for the CDF (black line) and its inverse (gray line). The samples from the proposal 
    distribution are shown as a light red histogram. Note that this histogram is not normalized, because the proposal distribution is
    not normalized. Subsequently, we use rejection sampling to get the light blue histogram. These samples match our target distribution well.
    }
\end{figure}

It takes about 4 times less iterations to accept $10^4$ samples.

\noindent The code:

\lstinputlisting{q1bc.py}

\noindent The output:

\input{q1bc_output.txt}
