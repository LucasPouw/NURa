\section{Exercise 1: Poisson distribution}

The Poisson distribution requires a division by a factorial. This factorial may overflow itself, or it may cause underflow when
dividing by it. Therefore, we calculate the factorial and the Poisson distribution in log-space. The logarithm of the factorial
is calculated with the function \texttt{log\_factorial()}. This function takes an array of integers as input, of which it finds 
the largest, which we call $k$. The function then calculates $\ln(1) + \ln(2) + ... + \ln(k - 1) + \ln(k)$ and stores the cumulative
sum in an array, such that the $n$th element of that array contains $\ln(n!)$. As such, the requested factorials in the input array
of integers can then be returned by indexing the cumulative array. This ensures that the minimum number of computations is done, given
an input array. However, the cumulative array does require more memory than strictly necessary.

The Poisson distribution is then calculated in log-space, while making sure that each computation returns a 32-bit float. Finally, the
answer is exponentiated to give the desired result.

\noindent The code:

\lstinputlisting{poisson.py}

\noindent The output:

\input{poisson_output.txt}
