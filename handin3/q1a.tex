\subsection{Q1a}

We find the maximum of $N(x)$ by minimizing $-N(x)$. Since this is a well-behaved, 1D function 
with only a single global minimum, the golden section method suffices. Starting from the initial
two-point bracket, $x \in [0, 5]$, we construct a three-point bracket using the algorithm from 
the lectures (i.e., propose the abscissa of a parabola and accept it if it is within a threshold,
otherwise, shift the points and try again.)

The prefactor is not needed in this computation, because the maximum will be located at the same
point regardless. This saves some operations during the minimization routine. We output the location
of the maximum and its value (now with the prefactor).

\noindent The output:

\input{q1a_output.txt}

\noindent The code:

\lstinputlisting{q1a.py}