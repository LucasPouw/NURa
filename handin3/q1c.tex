\subsection{Q1c}

\subsubsection{The Poisson likelihood}
The likelihood of our model is given by the product of Poisson distributions:
\begin{equation}
    \mathcal{L}(x | \vec{p}) = \prod_{i = 0}^{N_{\rm bins} - 1} \frac{\mu(x_{i}|\vec{p})^{y_{i}}\exp\left[-\mu(x_{i}|\vec{p})\right]}{y_{i}!} \, .
\end{equation}
The negative log-likelihood is then given by
\begin{equation}
    -\ln \mathcal{L}(x|\vec{p}) = -\sum_{i = 0}^{N_{\rm bins} - 1}\left[ y_{i} \ln(\mu(x_{i}|\vec{p})) - \mu(x_{i}|\vec{p}) - \ln(y_{i}!) \right] \, .
\end{equation}
We can take the limit to infinitesimal binwidth, such that there is either one or zero counts in each bin. 
This allows us to estimate the parameters without binning the data. In that case,
\begin{equation}
    -\ln \mathcal{L}(x|\vec{p}) = -\sum_{i = 0}^{N - 1} \ln(\mu(x_{i}|\vec{p})) + \int\mu(x|\vec{p}) \dd x \, ,
\end{equation}
where we have used $\ln(y_{i}!) = 0$ since $y_{i}$ is either zero or one.

Even though our target model is not normalized to 1 (instead, it integrates to $\langle N_{\rm sat} \rangle$), we can estimate the parameters for a normalized
model and then convert back to our target model by multiplying by $\langle N_{\rm sat} \rangle$, for plotting. In that case, we get $\int\mu(x|\vec{p}) \dd x = 1$, which
is independent of the parameters, so we can ignore it in the likelihood. Now that I write this down, I realize that otherwise we would have had
$\int\mu(x|\vec{p}) \dd x = \langle N_{\rm sat} \rangle$, which is still independent of the data, so the only time
save is removing the multiplication by $\langle N_{\rm sat} \rangle$ when calling the model...oh well.

Just like with the $\chi^2$, the scaling of our model with $n_{\rm halo}$ and the neglected terms have changed our minimized function from the actual likelihood
of the model. In order to calculate the actual minimum log-likelihood value, we bin our model (not scaling with $n_{\rm halo}$) and calculate the full equation:
\begin{equation}
    -\ln \mathcal{L}(x|\vec{p}) = -\sum_{i = 0}^{N_{\rm bins} - 1}\left[ y_{i} \ln(\mu(x_{i}|\vec{p})) - \mu(x_{i}|\vec{p}) - \ln(y_{i}!) \right] \, .
\end{equation}

We provide the minimum negative log-likelihood of both our normalized model and on the binned (unscaled) model (which is the one that was actually asked).
The results are plotted in the same way as our $\chi^2$ plot, with our binned model.

\begin{figure}[H]
    \centering
    \includegraphics[width=\textwidth]{plots/q1c.png}
    \caption{Histograms of the mean number of satellites per halo as a function of radial distance. The best fit $-\ln \mathcal{L}$ model is
    (red solid line) plotted on top of the binned data (black histogram). Even though the best-fit model is binned, the actual parameter estimation is
    done without binning.
    }
\end{figure}

\noindent The output:

\input{q1c_output.txt}

\noindent The code:

\lstinputlisting{q1c.py}