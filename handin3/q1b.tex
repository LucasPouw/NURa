\subsection{Q1b}

\subsubsection{Binning the data}

This problem spans orders of magnitude, so our bins should be log-spaced. Bins cause bias, so we choose them as thin as possible. Increasing
the number of bins increases the computation time, so we choose $N_{\rm bins} = 100$ to keep it managable. The range for these bins is fixed
at $[10^{-4}, 5]$ for all datasets. This means the number of empty bins is different for each file, but empty bins are also informative.

\subsubsection{Calculating $\langle N_{\rm sat} \rangle$}
$\langle N_{\rm sat} \rangle$ is given by the number of satellites divided by the number of haloes in each dataset (see output).

\subsubsection{Minimizing $\chi^{2}$}
Our minimization routine of choice is Downhill Simplex. The inital simplex is checked to be non-singular by computing its volume, using the method on 
Wikipedia: \url{https://en.wikipedia.org/wiki/Simplex#Volume}. Here, it says that the volume of a simplex is proportional to the determinant of the
Gram matrix $MM^{T}$, where $M$ is a matrix with columns of vectors that point from column 0 to column $k$. This determinant is calculated using the
matrix class from hand-in 1. Namely, the determinant of an LU-decomposed matrix is given by the product of its diagonal elements. Let $d$ be the
number of dimensions of our simplex (for us, $d=3$), then the volume of the simplex is given by
\begin{equation}
    V = \frac{1}{d!}\sqrt{|\det{MM^{T}}|}
\end{equation}
A degenerate simplex has zero volume, so we need a minimum initial volume threshold. We set this threshold at 0.1, because the parameters should change 
on the order of 0.1 to 1. A pyramid (which is a simplex/tetrahedron that is easy to work with) that has a volume 0.1 has base edge width of 
$(0.1 * 6 \sqrt{2})^{1/3} \approx 0.95$, which is indeed comparable to the expected dynamic range of our parameters.

Let's now look at the calculation of $\tilde{N_{i}} = 4 \pi \int_{x_{i}}^{x_{i+1}} n(x) x^{2} \dd x$. We first note that we need to calculate the
normalization $A = A(a,b,c)$ of the function $n(x) = n(x|a,b,c)$. As per hand-in 2, this normalization can be calculated from the requirement
\begin{equation}
    \iiint_{V} n(x) = 4 \pi \int n(x) x^{2} \dd x = \langle N_{\rm sat} \rangle \, ,
\end{equation}
which implies
\begin{equation}
    4 \pi A = \frac{1}{\int x^2 (x/b)^{a-3} \exp \left[- (x/b)^{c} \right]} \, .
\end{equation}
This is the quantity we calculate with the \texttt{normalization()} function in the code. Thus, when calculating $\tilde{N_{i}}$, we do not
require additional multiplication by $4 \pi$.

Now, there are a few things to watch when performing this integral. First, the $x^{a - 3}$ might give problems when $a < 3$ and $x \ll 1$. 
This is solved by incorporating the multiplication by $x^2$ in this term and writing $x^{a - 1}$. However, the Downhill Simplex method may
still step to regions where $a < 1$. Moreover, $b < 0$ also causes problems. In order to avoid these, we apply two priors in our $\chi^2$ 
model. We set $\chi^2 = \infty$ whenever $a < 1$ or $b < 0$. Finally, we use a midpoint Romberg integrator to be totally safe around $x = 0$.
All Romberg integrations have order $m = 6$, which keeps computation time low, while still providing sufficient precision.
For the Downhill Simplex, we set a target fractional accuracy of $10^{-8}$ and a maximum number of 100 iterations. 
The attained fractional accuracy is usually better.

To end, the minimum $\chi^2$ value should be scaled by $n_{\rm halo}$ to undo the scaling of our histogram. Namely, looking at the equation
for $\chi^2$:
\begin{equation}
    \chi^2 = \sum_{i = 0}^{N_{\rm bins} - 1} \frac{\left(y_{i} - \mu(x_{i} | \vec{p}) \right)^{2}}{\mu(x_{i} | \vec{p})}
\end{equation}
we see that division of our bins ($y_{i}$) and our model mean ($\mu(x_{i} | \vec{p})$) by $n_{\rm halo}$ gives a value of $\chi^2$ that is 
a factor $n_{\rm halo}$ too small. We provide the minimum $\chi^2$ and the reduced $\chi^2$ (for the number of DoF, see the answer to Q1d).

\begin{figure}[H]
    \centering
    \includegraphics[width=\textwidth]{plots/q1b.png}
    \caption{Histograms of the mean number of satellites per halo as a function of radial distance. The best fit $\chi^{2}$ model is
    (red solid line) plotted on top of the binned data (black histogram).
    }
\end{figure}

\noindent The output:

\input{q1b_output.txt}

\noindent The code:

\lstinputlisting{q1b.py}